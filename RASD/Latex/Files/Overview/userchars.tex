\begin{itemize}
	\item \textbf{{Guest}}: a guest is an unregistered user who can visit the system's homepage to learn about the opportunities offered by our service. It may choose to register by providing valid credentials or using an external login provider. Before registering to the system, and becoming a user, the guest cannot access any of the features offered by the system such as meeting attendance and user profile management.
	\item \textbf{{User}}: the user is a guest who has completed the registration process. In order to have full access to the functionalities offered by the system, it must have provided at least one default location and submitted its preference list. The user has a profile that can be edited by adding a phone number or additional emails. Furthermore the system gives each user the opportunity to create a list of contacts with which they can organize meetings more easily. The user can be part of groups, such as colleagues working on the same project, to organize frequent meetings with the same people. In addition, the system provides real time information about the best path to reach the next meeting and organizes user's breaks according to the schedule of the day.
	\item \textbf{{Administrator}}: the administrator is a user who has created a meeting after defining a title, a date, a location and the participants.  After the meeting has been created, the administrator can add or remove participants or designate other administrators. If a user responds to the participation request by proposing a rescheduling, the administrator can accept, decline or forward the reply to all members of the team; however the administrator has the authority to propose a change of date by itself. \newline
	It can upload files both before the meeting, to provide a complete program on the topics to all participants, and after the meeting, to share results with the team. These files can also be maintained and updated for subsequent meetings. In fact, the administrator has the opportunity to re-propose a meeting with the same people, keeping the topic and files unchanged, on a new date. \newline
	The system gives the administrator the power to monitor in real time whether a participant will arrive late - above a fixed threshold for privacy reasons - thanks to the system's ability to trace the position of the participants.
	\item \textbf{{System Manager}}: the system manager is an employee of \projectname~  who has the ability to maintain and update the system. It does not have an account that it can use to interface with other users but acts on the system to add new features or improve existing ones.
\end{itemize}
