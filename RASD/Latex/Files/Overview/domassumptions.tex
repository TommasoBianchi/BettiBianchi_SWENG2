\subsubsection{Domain Assumptions}

\newcounter{countDom}
\stepcounter{countDom}
\begin{description}

\item[D\thecountDom] All emails are valid according to RFC3696~\cite{RFC3696} standard.

\stepcounter{countDom}
\item[D\thecountDom] All emails belong to the user who enters them into the system.

\stepcounter{countDom}
\item[D\thecountDom] External login systems don’t fail and always provide a valid email to identify the registered user.

\stepcounter{countDom}
\item[D\thecountDom] All users complete their profile right after the registration phase \\ [0.1cm]
After registration, users must complete the profile to be able to use the system. They have to add a nickname, which must be unique, complete the preference list about travel means and add at least one default location. This is needed to prevent inconsistencies in the system (e.g. the system cannot infer the location of a user who has never participated to a meeting if it has not a default position).

\stepcounter{countDom}
\item[D\thecountDom] Each user has at least one default location \\ [0.1cm]
The default location is where a user is considered to be located during the day. The system will compute the time when the user should start the travel to a meeting from the default location. Only after that time, the system will use the user's position to provide real-time information. This in needed to take into account the fact that, for example, someone may want to go back to his office between a meeting in the morning and one in the afternoon.

\stepcounter{countDom}
\item[D\thecountDom]  User can have lunch everywhere. \\ [0.1cm]
This assumption allows the system to consider that a user can always have a break without any necessary movement.

\stepcounter{countDom}
\item[D\thecountDom]  Each user has a preference list.

\stepcounter{countDom}
\item[D\thecountDom]  All users always have a position. \\ [0.1cm]
This is needed for the system to provide the user real-time information about the trip.

\stepcounter{countDom}
\item[D\thecountDom]  External shortest path provider is always able to retrieve a path between any two locations.

\stepcounter{countDom}
\item[D\thecountDom] Each user is always able to communicate with our servers.

\stepcounter{countDom}
\item[D\thecountDom] The system does not differentiate between a travel mean that is shared and one that is owned. \\ [0.1cm]
A shared car can be considered exactly equal to a owned car because it takes the same amount of time to complete a certain path and it would be in the same position in the preference list. This assumption is taken to avoid the complexity to keep track of things like where an owned car is parked and how to get there. It also avoids the presence of mixed travel means in the system (i.e. take a bus up to your car and then drive it), that would be difficult to deal with, especially from the constraints point of view.
\normalsize

\stepcounter{countDom}
\item[D\thecountDom] The system treats taxis as a driving travel mean and not as public transportation. \\ [0.1cm]
This assumption is supported by the fact that taxis will take the same amount of time as a car to complete the path and, exactly like cars, can be considered to be always present in the same location of the user. This will also simplify the development process as most of external shortest path providers available on the market take the same assumption.

\stepcounter{countDom}
\item[D\thecountDom] If a user accepts the invitation to a meeting, then he really attends to it.  \\ [0.1cm]
This assumption is needed to always correctly infer the location of a user. In fact, without this assumption a user could not attend a meeting and therefore the travel indications to go from that meeting to the following one could be completely wrong.
\end{description}