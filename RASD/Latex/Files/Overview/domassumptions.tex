\subsubsection{Domain Assumptions}

\newcounter{countDom}
\stepcounter{countDom}
\begin{description}

\item[D\thecountDom] All emails are valid according to RFC3696~\cite{RFC3696} standard
\stepcounter{countDom}
\item[D\thecountDom] All emails belong to the user who enters them into the system

\stepcounter{countDom}
\item[D\thecountDom] External login systems don’t fail and always provide a valid email to identify the registered user

\stepcounter{countDom}
\item[D\thecountDom] All users complete their profile right after the registration phase \\ [0.1cm]
After registration, users must complete the profile to get the best functionality from the system. They have to add a nickname, which must be unique, and complete the preference list about travel means

\stepcounter{countDom}
\item[D\thecountDom] Each user has at least one default location \\ [0.1cm]
This default position is considered where the users should be located during the day. The system will compute the time when the user should start the travel to a meeting from the default location. Only after that time, the system will trace the user's position to provide real-time information

\stepcounter{countDom}
\item[D\thecountDom]  User can have lunch everywhere \\ [0.1cm]
This assumption allows the system to consider that you can always have a break without any necessary movement

\stepcounter{countDom}
\item[D\thecountDom]  Each user has a preference list

\stepcounter{countDom}
\item[D\thecountDom]  All users always have a position \\ [0.1cm]
This assumption is useful when the system must provide the user real time informations about the trip

\stepcounter{countDom}
\item[D\thecountDom]  External shortest path provider is always able to retrieve a path between any two locations

\stepcounter{countDom}
\item[D\thecountDom] Each user is always able to communicate with our servers

\stepcounter{countDom}
\item[D\thecountDom] The system does not differentiate between a travel mean that is shared and one that is owned \\ [0.1cm]
A shared car can be considered exactly equal to a owned car because it takes the same amount of time to complete a certain path and it would be in the same position in the preference list
\normalsize

\stepcounter{countDom}
\item[D\thecountDom] The system treats the taxis as a driving travel mean and not as public transportation \\ [0.1cm]
This assumption is supported by the fact that taxis will take the same amount of time as a car to complete the path and, exactly like cars, can be considered to be always present in the same location of the user
\end{description}