\newcounter{countScenarios}
\stepcounter{countScenarios}

\subsubsection{Scenario \thecountScenarios }
\stepcounter{countScenarios}
--- guest registration  to the system + edit user's profile adding another email, phone number, website, social networks, company, position \\

Alice is a manager of a large company and mother of two children. She is always on the go, both to attend meetings all around the city and to bring her sons around. Up until now, she has tried her best to manage all this with her old style agenda, with the only result of being often late and under constant stress. Yesterday a colleague told her about a fantastic new service, \projectname~, that seems to be a perfect fit for her needs, so she immediately decides to try it out. Now she has a few minutes of break, so she decides to register to the system and customize her profile. \underline{She adds a profile picture, her phone number and her Facebook account.}

- adding a user to your contact after having meet him in a meeting \\
- after a meeting create a group with the team to organize meeting more frequently \\


\subsubsection{Scenario \thecountScenarios }
\stepcounter{countScenarios}
--- create a meeting with other users, some of them taken from your contacts\\

Bob is the CEO of a new, innovative startup, still substantially unheard of. His main concern now is to attract potential investors by describing his idea to as many of them as possible, so he is constantly around to attend meetings and to meet new people. To reach his goal, he has used \projectname~ throughout last month and now he has a good number of contacts saved. Now he feels ready to try to arrange a meeting with the executives of a large company that he thinks should be interested in what he does; thereby he create a meeting on \projectname~ inviting \st{both} them and a couple of his contacts, who he would like to get into his team if things were to get going.

- administrator add files before and after and they could be seen from the team or use chat\\
- administrator see who is late, after a threshold\\


\subsubsection{Scenario \thecountScenarios }
\stepcounter{countScenarios}
--- user try to reschedule a meeting and the admin send the request to other participants\\

Charlie is a freelance web developer, so he has to frequently meet with all the different companies that hire him. To do so, he uses a popular service named \projectname~, which allows him to always avoid to arrange meetings in a way that will make impossible for him to attend them all. Today, he received an invitation from Dave, social media manager of a company for which Charlie maintains a website since years. All Dave's team has been invited, since they have to discuss some fairly substantial improvement on the graphics. The proposed date for the meeting is Thursday at 7 PM, but Charlie is a commuter so for him it is a problem as he has to take a train. Therefore he propose to reschedule the meeting on the subsequent day, at 4 PM, and Dave forwards this request to his team as he knows that it may be a problem for some of them to work till late also on Friday.


\subsubsection{Scenario \thecountScenarios }
\stepcounter{countScenarios}
Eve is a fashion store manager in Milan. During the Milan Fashion Week her agenda is full of meeting with brands, models and stylists. One of her colleague some weeks earlier suggested Eve to use \projectname~ to manage her schedule and Eve, although initially a little skeptical, decides to try this new service and downloads the application. As hers nutritionist recommend, Eve sets a small break of 15 minutes in the afternoon between 4.00 PM and 6.00 PM to have a healthy snack and avoid stress caused by daily travels. In hers \projectname~'s agenda she has a meeting up to 5.20 PM and she has been invited to an appointment with a stylist at 6.00 PM on the other side of Milan. The system computes that, due to traffic jam, Eve should leave immediately after the previous meeting. This would prevent Eve doing her break so a warning is created to signal it. Eve, who had completely forgotten this chance, thanks \projectname~ and decides to reschedule the second meeting half an hour later, with the button on the application. The stylist accepts the proposal and Eve manages to have her break and stay fit.
\\
\\
\\
\\
- user has been invited to a meeting but a warning is generated because this meeting would block all the possible time slots for a break, choose something\\
- the system informs the user that he has to take a break, have a break (maybe lunch) with other users and after that user add some of the others to his contacts (from the lunch you know someone and add him to your contacts)(maybe similar to the second)\\


\subsubsection{Scenario \thecountScenarios }
\stepcounter{countScenarios}
--- the system calculate when it should start to give real time indication to the user, process all possible path and select the best one\\

Freddie is a pretty busy university professor. His \projectname~ schedule is always full of meetings everywhere in London. He's deeply aware of today's pollution problems, so he always tries to use cars as less as possible. He's also somehow uncomfortable in getting by bike to formal meetings, so his preference list looks as follows:
\begin{enumerate}
\item Walking
\item Public Transportation
\item Driving
\end{enumerate}
He has also setup some constraints, in particular because he's getting old and so he cannot walk for long distances or under a heavy rain. Moreover, he believes it is not wise to travel alone after the sunset so, in that case, he resorts to the car even if unwillingly. Here there are his constraints:
\begin{itemize}
\item Distance > 2 km - Walking
\item Weather is rainy - Walking
\item Time > 18 - Walking
\item Time > 18 - Public Transportation
\end{itemize}
Now he has just accepted a meeting for Tuesday morning on 11 AM and the system has to calculate a few things in order to manage it correctly. Initially, all the travel times from the previous meeting and from his default location (that happens to be his office at 11 AM) are calculated, giving the following results:
\begin{enumerate}
\item Walking		from previous meeting 21 minutes, from office 53 minutes 
\item Public Transportation		from previous meeting 11 minutes, from office 23 minutes
\item Driving		from previous meeting 8 minutes, from office 15 minutes
\end{enumerate}
Then, all constraints are taken into account, eliminating the possibility to go by foot from the office as it is longer than 2 km. The previous meeting ends at 10 AM, so no warning has to be created. The selected travel means are walking from the previous meeting and public transportation from the office; as the slowest of the two takes 23 minutes, this value is saved for later use. On the day of the meeting, starting from a little before than 23 minutes before 11 AM, the real-time position of Freddie is used to calculate the distance to the meeting location and, if necessary, to notify him that he needs to leave in order not to be late.


\subsubsection{Scenario \thecountScenarios }
\stepcounter{countScenarios}
George is a manager of a \st{constructor} construction company working in Modena and its province. After the 2012 earthquake lots of monuments and houses need to be reconstructed so his agenda is full of meeting with clients and architects. His company has decided to use \projectname~ to organize all the meetings. George's secretary has create a meeting with some architects and engineers and has nominated George administrator. \\
After the meeting, George decides to make a summary of all the achievements reached so far and to upload that on the meeting's homepage. In this way the document can be seen by all the participants. The team has to organize another meeting to complete its goals and George, as meeting administrator, uses the \projectname~ functionality that allows to create a new meeting with the same participants, keeping uploaded files. The following meeting will be in the same location, two weeks after. The system automatically invites all the participants to the second meeting. This is a meeting exactly like others, so each participant has to accept, decline or propose a rescheduling.\\

