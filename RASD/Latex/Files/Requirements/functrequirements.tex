\newcounter{countReq}
\setcounter{countReq}{1}

\begin{description}
\item[G\thecountReq] Allow someone to visit the homepage of the system and to register himself providing a valid email, a password and a unique nickname (or via a third-party login system such as Google+).
\stepcounter{countReq}

\begin{itemize}
\item~[R] Each user must provide an email that is not already present in the system.
\item~[R] Each user must provide a nickname that is not already present in the system.
\end{itemize}

\begin{itemize}
\item~[D] All emails are valid according to RFC3696~\cite{RFC3696} standard.
\item~[D] All emails belong to the user who enters them into the system.
\item~[D] External login systems don’t fail and always provide a valid email to identify the registered user.
\item~[D] All users complete their profile right after the registration phase.
\end{itemize}

\item[G\thecountReq] Users can log into the system.
\stepcounter{countReq}

\begin{itemize}
\item~[R] Users have to be registered into the system before logging in.
\item~[R] Users have to provide an existing email or nickname and the associated password in order to log in; in alternative they can use a supported external login system.
\end{itemize}

\begin{itemize}
\item~[D] External login systems don’t fail and always provide a valid email to identify the registered user.
\end{itemize}

\item[G\thecountReq] Allow a user to edit all information in its profile (e.g. displayed name, phone number, company, website, social accounts).
\stepcounter{countReq}

\begin{itemize}
\item~[R] Each user must provide an email that is not already present in the system.
\item~[R] Each user must provide a nickname that is not already present in the system.
\item~[R] Users cannot have meetings while their status is set to auto-decline.
\item~[R] A user cannot have different default locations sharing the start time.
\item~[R] A user must be logged into the system to perform any action except registering and logging in.
\end{itemize}

\item[G\thecountReq] Allow a user to create a meeting and to invite other users (at least one) to attend it.
\stepcounter{countReq}

\begin{itemize}
\item~[R] A user must be logged into the system to perform any action except registering and logging in.
\item~[R] Each meeting has at least two participants.
\item~[R] Each meeting has at least one administrator.
\item~[R] Each meeting has a title, a date and a location.
\item~[R] Each participant in a meeting can access shared files and the chat.
\item~[R] Users participate in a meeting if and only if they accept the invitation.
\item~[R] Users do not participate in a meeting if they decline the invitation.
\end{itemize}

\item[G\thecountReq] Create a warning each time it is not possible to reach a meeting location from the previous one.
\stepcounter{countReq}

\item[G\thecountReq] Allow a user to specify flexible breaks during the day.
\stepcounter{countReq}

\begin{itemize}
\item~[R] A user must be logged into the system to perform any action except registering and logging in.
\item~[R] The system suggests you a time, according to your settings, to have a break such that no meeting overlaps with it; if no time slot is valid, a warning is generated.
\end{itemize}

\begin{itemize}
\item~[D] Users can have lunch everywhere.
\end{itemize}

\item[G\thecountReq] Manage users’ travels between subsequent meetings.
\stepcounter{countReq}

\begin{itemize}
\item~[R] A user must be logged into the system to perform any action except registering and logging in.
\item~[R] The travel mean suggested by the system is always the first in the weighted preference list that satisfied all the constraints; if no travel mean satisfied all the constraints than the system suggests the fastest one.
\end{itemize}

\begin{itemize}
\item~[D] Each user has a preference list.
\item~[D] All users always have a position.
\item~[D] External shortest path provider is always able to retrieve a path between any two locations.
\item~[D] Each user is always able to communicate with our servers.
\item~[D] The system does not differentiate between a travel mean that is shared and one that is owned.
\item~[D] The system treats the taxis as a driving travel mean and not as public transportation.
\end{itemize}
\end{description}