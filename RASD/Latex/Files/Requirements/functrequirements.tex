\newcounter{countGoal}
\setcounter{countGoal}{1}

\newcounter{countReq}
\newcommand{\reqNum}{\stepcounter{countReq}\thecountReq}

\begin{description}
\item[G\thecountGoal] Allow someone to visit the homepage of the system and to register himself providing a valid email, a password and a unique nickname. As an alternative, an external login provider, such as Google+, can be used.
\stepcounter{countGoal}

\begin{itemize}
\item~[R\reqNum] Each user must provide an email that is not already present in the system.
\item~[R\reqNum] Each user must provide a nickname that is not already present in the system.
\end{itemize}

\begin{itemize}
\item~[D1] All emails are valid according to RFC3696~\cite{RFC3696} standard.
\item~[D2] All emails belong to the user who enters them into the system.
\item~[D3] External login systems don’t fail and always provide a valid email to identify the registered user.
\item~[D4] All users complete their profile right after the registration phase.
\end{itemize}

\item[G\thecountGoal] Users can log into the system.
\stepcounter{countGoal}

\begin{itemize}
\item~[R\reqNum] Users have to be registered into the system before logging in.
\item~[R\reqNum] Users have to provide an existing email or nickname and the associated password in order to log in; in alternative they can use a supported external login system.
\end{itemize}

\begin{itemize}
\item~[D3] External login systems don’t fail and always provide a valid email to identify the registered user.
\end{itemize}

\item[G\thecountGoal] Allow a user to visit its profile and to see a detailed schedule of any day containing all the meetings he is attending and all the travels the system has planned him.
\stepcounter{countGoal}

\begin{itemize}
\item~[R\reqNum] A user must be logged into the system to perform any action except registering and logging in.
\end{itemize}

%\begin{itemize}
%\end{itemize}

\item[G\thecountGoal] Allow a user to edit all information in its profile (e.g. displayed name, phone number, company, website, social accounts).
\stepcounter{countGoal}

\begin{itemize}
\item~[R1] Each user must provide an email that is not already present in the system.
\item~[R2] Each user must provide a nickname that is not already present in the system.
\item~[R\reqNum] Users cannot have meetings while their status is set to auto-decline.
\item~[R\reqNum] A user cannot have different default locations sharing the start time.
\item~[R\reqNum] Time travel between subsequent default locations should be less than the difference between their start time.
\item~[R5] A user must be logged into the system to perform any action except registering and logging in.
\end{itemize}

\item[G\thecountGoal] Allow a user to create a meeting and to invite other users to attend it.
\stepcounter{countGoal}

\begin{itemize}
\item~[R5] A user must be logged into the system to perform any action except registering and logging in.
\item~[R\reqNum] Each meeting has at least two participants.
\item~[R\reqNum] Each meeting has at least one administrator.
\item~[R\reqNum] Each meeting has a title, a date and a location.
\item~[R\reqNum] Each participant in a meeting can access shared files and the chat.
\item~[R\reqNum] Users participate in a meeting if and only if they accept the invitation.
\item~[R\reqNum] Users do not participate in a meeting if they decline the invitation.
\item~[R\reqNum] Users can write in the chat of a meeting if and only if they have received and accepted an invitation to it.
\end{itemize}

\item[G\thecountGoal]For each meeting there is a warning iff the meeting is inconsistent.
\stepcounter{countGoal}

\begin{itemize}
\item~[R11] Each meeting has a title, a date and a location.
\end{itemize}

\begin{itemize}
\item~[D13] If a user accepts the invitation to a meeting, then he really attends to it.
\end{itemize}

\item[G\thecountGoal] Allow a user to specify flexible breaks during the day.
\stepcounter{countGoal}

\begin{itemize}
\item~[R5] A user must be logged into the system to perform any action except registering and logging in.
\item~[R\reqNum] The system suggests you a time, according to your settings, to have a break such that no meeting overlaps with it; if no time slot is valid, a warning is generated.
\end{itemize}

\begin{itemize}
\item~[D6] Users can take a break everywhere.
\end{itemize}

\item[G\thecountGoal] Manage users’ travels between subsequent meetings, suggesting the best mobility option according to their preference list.
\stepcounter{countGoal}

\begin{itemize}
\item~[R5] A user must be logged into the system to perform any action except registering and logging in.
\item~[R\reqNum] At least one travel mean is available in the preference list.
\item~[R\reqNum] The travel mean suggested by the system is always the first in the weighted preference list that satisfied all the constraints; if no travel mean satisfied all the constraints than the system suggests the fastest one.
\end{itemize}

\begin{itemize}
\item~[D5] Each user has at least one default location.
\item~[D7] Each user has a preference list.
\item~[D8] All users always have a position.
\item~[D9] External shortest path provider is always able to retrieve a path between any two locations.
\item~[D10] Each user is always able to communicate with our servers.
\item~[D11] The system does not differentiate between a travel mean that is shared and one that is owned.
\item~[D12] The system treats the taxis as a driving travel mean and not as public transportation.
\item~[D13] If a user accepts the invitation to a meeting, then he really attends to it.
\end{itemize}
\end{description}
