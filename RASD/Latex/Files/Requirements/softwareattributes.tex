\begin{itemize}
	\item \textbf{Reliability}: The reliability of the system is a crucial point in the development process since it is not possible to provide a professional organization service that contains failures. System managers should ensure that the system \st{has to be} will be fully tested and error-free before entering the market. All of this, combined with a high maintenance speed in case of unexpected errors, is the starting point to provide high availability.
	\item \textbf{Availability}: Since the system must guarantee a complete service to the user, it must ensure 24/7 operation. Especially in view of a future global expansion, the service will have to offer all the functions 24 hours a day and for this reason it will be able to afford only a few exceptions from the 24/7 target. \underline{By Tommy}: In the first implementation, however, as the system will operate only in a single country, availability requirements may be a little relaxed during the night time.
	\item \textbf{Security}: During registration phase or through subsequent changes, users will provide the system confidential and sensitive informations that must be properly protected to ensure their privacy. In addition, the system must also be \st{constructed} designed in such a way that a user cannot access \st{to} another user's schedule.
	\item \textbf{Maintainability}: Maintainability is essential because this system is created to allow future expansion and integration of new functionalities. This is possible only if the software has excellent characteristics in terms of testing \st{(above xxxx)}, understandability and code modifiability (given by a great structural layout) .
	\item \textbf{Portability}: The system is structured to be fully cross-platform. Its functionalities must be accessible by both web interfaces and smartphone applications.
\end{itemize}