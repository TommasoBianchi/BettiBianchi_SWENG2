The RASD is organized into 5 main sections:

\begin{enumerate}
\item {\textbf{\nameref{sect:introduction}}}: this section contains a compact and high level view on the problem and on the goals of the system to be implemented in order to address it; it also contains a glossary with the definitions of the main terms we will refer to in the rest of the document.
\item {\textbf{\nameref{sect:overview}}}: this section  describes the general factors that affect the product and its requirements, without stating specific requirements. Its contents are still at a high level of abstraction and should provide a background to understand more easily the following sections.
\item {\textbf{\nameref{sect:requirements}}}: this section contains all the software requirements at a level of detail sufficient to enable designers to design a system to satisfy those requirements, and testers to test that the system satisfies those requirements. Each requirement is stated in relation to goals and domain assumptions.
\item {\textbf{\nameref{sect:alloy}}}: this section contains all the code of a formal representation of the system using the alloy language and the results of the analysis of requirements in such a setting.
\item {\textbf{\nameref{sect:effort}}}: this section contains the amount of time spent by each of the author of this document working on it.
\end{enumerate}