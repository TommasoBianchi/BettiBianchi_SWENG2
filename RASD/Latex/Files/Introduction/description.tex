\subsubsection{Definitions}

The following list contains all the main terms we will refer to in the rest of the document, together with a brief explanation of their meaning in this context.

\begin{itemize}
\item \textbf{User}: anyone that is registered into the system.
\item \textbf{Nickname}: an alphanumerical string uniquely identifying a user.
\item \textbf{User’s Credentials}: either the pair <email, password>, the pair <nickname, password> or the token, returned by an external login provider; they do identify a precise user and have to be provided in order to log into the system.
\item \textbf{Guest}: anyone that is not registered into the system.
\item \textbf{Contact}: users will have the possibility to save other users as their contacts to find them more easily when needed.
\item \textbf{Break}: a time slot in which a user does not want to have meetings; its exact starting and ending time may vary in a fixed interval to achieve more flexibility.
\item \textbf{Group}: a collection of users; it may be used to simplify the creation of recurring meetings among the same people.
\item \textbf{Status}: a flag indicating if a user is in normal or auto-decline status; in the latter case, all incoming request of meeting will automatically be declined.

\item \textbf{Position}: the real-time latitude and longitude representing where a user is.
\item \textbf{Location}: the latitude and longitude of a meeting or a user as inferred by the information known a priori by the system (e.g. the location inserted when a meeting is created).
\item \textbf{Default Location}: users can provide default locations to indicate where they are when not at a meeting (e.g. the office); they are represented by a location and a start time and they are assumed to be valid until the start time of another default location.

\item \textbf{Meeting}: an event involving at least two users taking place in a specific location and in a specific date.
\item \textbf{Category}: a tag representing the type of a meeting (e.g. work, family).
\item \textbf{Administrator}: a user that can manage most of the settings and data regarding a meeting.
\item \textbf{Title}: a string to shortly describe the purpose of a meeting.
\item \textbf{Abstract}: a text to give information about a meeting (e.g. agenda of the day).
\item \textbf{Participant}: a user that has accepted an invitation to a meeting.
\item \textbf{Team}: the collection of all participants of a given meeting.
\item \textbf{Instant Meeting}: a special type of meeting that has only one participant and the same start and end time. It may be useful for organizing everything that has not a real duration (such as picking up someone or something) or that do not require interaction with other users. For all other aspects it is a meeting like normal ones, and in particular it can be incompatible and raise warnings.

\item \textbf{Incompatible Meeting}: a meeting that overlaps with another one, that cannot be reached from the previous one in time or that prevents the correct placement of a break.
\item \textbf{Warning}: a message displayed to a user indicating that he has been invited to an incompatible meeting.

\item \textbf{Travel mean}: a way to travel between different locations.
\item \textbf{Walking}: a travel mean in which you always go by foot.
\item \textbf{Biking}: a travel mean in which you use your bike or a bike sharing service.
\item \textbf{Driving}: a travel mean in which you use your car, a car sharing service or a taxi.
\item \textbf{Public Transportation}: a travel mean in which you use metros, trains, buses and/or trams.

\item \textbf{Constraint}: a restriction on when the system is allowed to suggest a specific travel mean; it will be composed as \texttt{subject operator value - target} (e.g. \texttt{'weather' 'is' 'rainy' - 'walking'}, \texttt{'time' '>' '21' - 'public transportation'}) and works as follows: if the operator applied to the observed state of the subject and the value yields true, then the target cannot be selected as suggested travel mean.
\item \textbf{Constraint Subject}: a term representing an element of the world which the constraint refers to (e.g. the weather, the time of the day); it has to be fully observable by the system.
\item \textbf{Constraint Operator}: a binary function from a pair of subject's states to a boolean value.
\item \textbf{Constraint Value}: one of the states that characterize the subject (e.g. rainy, an integer).
\item \textbf{Constraint Target}: the travel mean affected by the constraint.

\item \textbf{Preference List}: an ordered list of the travel means where the highest ones are those preferred by a user; travel means that do not appear on this list are considered as non-usable by the user.
\item \textbf{Weighted Preference List}: a sorted list of travel times calculated from the travel means of the preference list where (i * k) minutes have been added to the i\textsuperscript{th} element; it is a way not to suggest a user very long travels with their preferred travel means if a much shorter one is possible (e.g. if a user can walk for 4 hours or drive for half an hour suggest to drive even if walking is higher in the priority list than driving).
\end{itemize}

\subsubsection{Acronyms}

\begin{itemize}
\item~\textbf{RASD}: Requirement Analysis and Specification Document (the present document).
\item~\textbf{UML}: Unified Modeling Language (a specification defining a graphical language for visualizing, specifying, constructing, and documenting the artifacts of distributed object systems).
 
\end{itemize}

\subsubsection{Abbreviations}

\begin{itemize}
\item~\textbf{[Gn]}: n\textsuperscript{th} goal.
\item~\textbf{[Dn]}: n\textsuperscript{th} domain assumption.
\item~\textbf{[Rn]}: n\textsuperscript{th} requirement.
\end{itemize}