\subsubsection{Definitions}

Since the domain of our system remains the same, the terminology that we will use throughout this document is still the one we defined in the RASD document.

\subsubsection{Acronyms}

\begin{itemize}
\item \textbf{Implementation \& Test Document (ITD)}: the present document.
\item \textbf{Object Relational Mapping (ORM)}: a programming technique for converting data between incompatible type systems using object-oriented programming languages.
\item \textbf{Ruby on Rails (RoR)}: a server-side web application framework written in Ruby under the MIT License.
%\item \textbf{Unified Modelling Language (UML)}: a specification defining a graphical language for visualizing, specifying, constructing, and documenting the artifacts of distributed object systems.
%\item \textbf{Commercial Off-the-Shelf (COTS)}: software and services that are built and delivered usually from a third party vendor.
%\item \textbf{Model-View-Controller (MVC)}: a design pattern.
%\item \textbf{Uniform Resource Locator (URL)}: a reference to a web resource that specifies its location on a computer network and a mechanism for retrieving it.
%\item \textbf{Application Programming Interface (API)}: a set of subroutine definitions, protocols, and tools for building application software.
%\item \textbf{User Interface (UI)}: the space where interactions between humans and machines occur.
%\item \textbf{User Experience (UX)}: a person’s perceptions of system aspects such as utility, ease of use and efficiency.
%\item \textbf{Global Positioning System (GPS)}: a global navigation satellite system that provides geolocation and time information to a GPS receiver anywhere on Earth.
\end{itemize}

%\subsubsection{Abbreviations}

%\begin{itemize}
%\end{itemize}
