\begin{itemize}
\item~\textbf{Ruby on Rails is an opinionated framework}: this means it guides you into their way of doing things and promotes the best standards and practices of web development. The central pillar of the Rails philosophy is the DRY (Don’t Repeat Yourself) principle that ensures a clear separation of concerns and maintainability of your application. The framework embraces the principle of ‘convention over configuration’, according to which Rails defaults to a set of conventions that specify the best way of doing many things.

\item~\textbf{Speed of Development}: Rails well-developed system of modules, generator scripts, and an efficient package management system allow scaffolding a complex application in just a few commands. We can achieve rapid application development thanks to the expressive and concise nature of Ruby language, and also to dozens of open-source libraries for just about any purpose, which the Ruby community calls ‘gems’. As an added bonus, Rails ships with a default ORM system (ActiveRecord), which helps developers quickly put application and data logic together and deploy a fully functional prototype.

\item~\textbf{Rails is an open-source web framework supported by an active community}: Rails developers are interested in the constant improvement of the code base and incorporation of new functionalities. As a result, with Rails, there is no need to reinvent the wheel in your projects. RoR’s ecosystem contains many “gems”, i.e. pieces of software that can be incorporated into your project. Ruby community takes care of that. Almost any functionality you might need for your web project has already been created. A vibrant community that runs Rails also ensures that the framework is regularly updated, issues are fixed, and security is kept up-to-date with the best industry standards. Also, a big active community means a lot of online material for learning and fast and competent answer to all the problems that may arise during the development.
\end{itemize}