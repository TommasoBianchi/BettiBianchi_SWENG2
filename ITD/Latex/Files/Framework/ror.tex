Ruby on Rails, or Rails, is a server-side web application framework written in Ruby under the MIT License. Rails is a model-view-controller (MVC) framework, providing default structures for a database, a web service, and web pages. It encourages and facilitates the use of web standards such as JSON or XML for data transfer, and HTML, CSS and JavaScript for display and user interfacing. In addition to MVC, Rails emphasizes the use of other well-known software engineering patterns and paradigms, including convention over configuration (CoC), don't repeat yourself (DRY), and the active record pattern.

We have chosen to implement the \projectname~ application using the Rails framework and the Ruby language because of its very fast development cycles that permitted us to have a first working prototype in days, and from there to rapidly build up all the functionalities that we managed to include. Another reason for choosing Rails is the big and very active community it has, that helped us finding very easily answers for most of the problems that we had to face during the development. A last, secondary motivation is the tight integration with Heroku, a cloud Platform-as-a-service where we have setup an automatic deploy from our Github repository and that is currently hosting an online working version of the application to showcase \projectname~ more easily.