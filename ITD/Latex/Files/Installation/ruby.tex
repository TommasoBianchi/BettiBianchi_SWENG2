You can find full installation instruction for the Ruby language \href{https://www.ruby-lang.org/en/documentation/installation/}{here}. Below there is a short summary for the most used operating systems.

\begin{itemize}
\item~\textbf{Linux rvm (tested and suggested)}

RVM (Ruby Version Manager) is a command-line tool which allows you to easily install, manage, and work with multiple ruby environments from interpreters to sets of gems. To install it open a terminal and copy-paste the following code: \\
\begin{commandshell}
sudo apt-get install libgdbm-dev libncurses5-dev automake libtool bison libffi-dev curl
gpg --keyserver hkp://keys.gnupg.net --recv-keys 409B6B1796C275462A1703113804BB82D39DC0E3
curl -sSL https://get.rvm.io | bash -s stable
source ~/.rvm/scripts/rvm
rvm install 2.5.0
rvm use 2.5.0 --default
gem install bundler
\end{commandshell}

\item~\textbf{Linux (Debian or Ubuntu)}

Open a console and type in \begin{commandshell}sudo apt-get install ruby-full\end{commandshell}

\item~\textbf{Windows}

Download the installer from \href{http://railsinstaller.org/en}{here}, then run it.

\item~\textbf{MacOSX (Using Homebrew)}

Open a console and type \begin{commandshell}brew install ruby\end{commandshell}

\item~\textbf{MacOSX (Installer)}

Download the installer from \href{http://railsinstaller.org/en}{here}, then run it.
\end{itemize}