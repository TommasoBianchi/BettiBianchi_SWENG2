The work of the algorithm is divided in two main functions: \texttt{insert\_meeting} to insert a new meeting into the user's schedule and \texttt{best\_travel} to compute the best travel between two locations with a given travel mean, taking into account the user's preference list and constraints. To simplify the execution of the algorithm and to speed it up, inconsistent meetings will be treated as non-existent, so that we can avoid to consider them when calculating travels; this is a fairly strong assumption, but it is needed to avoid all sort of problems tied to overlapping meetings, where it is unclear how to travel between them. \\
\\
The first function, \texttt{insert\_meeting}, does its job following this steps:
\begin{enumerate}
\item Find all the meetings overlapping with the one that we are inserting and mark them as inconsistent because they are clearly in conflict; if at least a meeting is found, mark also the new one as inconsistent and terminate the function. While doing this, keep track of all overlaps and store them; we will need this information to update meetings' consistency after a conflict is solved.
\item Try to compute a travel for arriving and for leaving the newly inserted meeting taking into account the previous meeting, the following one and the default locations; if this is not possible, terminate the function marking the new meeting and the conflicting one as inconsistent.
\item Adjust the effective time of all the flexible breaks overlapping with the new meeting; if this is not possible, mark the break as not doable. While doing this, keep track of all overlaps and store them; we will need this information to update breaks' doability after a conflict is solved.
\item Link the travels to the meeting and store everything, then terminate successfully.
\end{enumerate}
\\
The second function, \texttt{best\_travel}, does its job following this steps:
\begin{enumerate}
\item Fetch a path from the External Shortest Path Provider for each travel means in the user's preference list.
\item Discard all the paths that are not compatible with the user's constraints.
\item Build the wighted preference list with the path calculated above.
\item Return the path corresponding to the first item of the weighted preference list.
\end{enumerate}