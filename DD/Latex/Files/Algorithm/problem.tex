The most critical algorithmic problem to solve in \projectname~ system is the insertion of a new meeting in a user's schedule. This is not as simple as it could seem because for every new meeting in a schedule a lot of things need to be taken into account and, possibly, recalculated and updated, such as whether the other meetings remain consistent, where flexible breaks can be placed, how to travel between subsequent meetings and so on. Some important choices have to be made, such as if to travel from a meeting to a default location and then to the next meeting or directly from one meeting to the other. In addition to that, this algorithm is likely to run quite often, as our users are likely to have a few meetings each day and possibly many more invitations, plus as many flexible breaks as they want. Our implementation must therefore take all of this into account and reach a compromise between efficiency and precision, in order to build meaningful schedules without stressing our system too much.