The RASD is organized into 5 main sections:

\begin{enumerate}
\item {\textbf{\nameref{sect:introduction}}}: this section contains an overview on the purpose and the scope of this document, toghether with a glossary with the definitions of the main terms we will refer to in the followings.
\item {\textbf{\nameref{sect:architecture}}}: this section contains a description of the architecture proposed for the development of the \projectname~ project, described at different levels of abstractions mainly by using UML diagrams. This should be the base of all the implementation efforts and serve as a model to which the implementation should aim at.
\item {\textbf{\nameref{sect:ui}}}: this section contains a more detailed description of the user interface of the system that the one briefly presented in the RASD document. This is achieved by using UX diagrams that describe the entirety of the interaction between the system and a user, toghether with a selection of mockups that should cover the most important pages and sketch the style of what will be our complete UI implementation.
\item {\textbf{\nameref{sect:algorithm}}}: this section contains the description and the pseudocode of an algorithm proposed to solve the problem of scheduling new meetings, a core functionality in our system and one of the most delicate.
\item {\textbf{\nameref{sect:implementation}}}: this section contains a description of the plan we will use to coordinate the implementation efforts, toghether with some indications on how to test the various components of the system and their interaction.
\item {\textbf{\nameref{sect:appendix}}}: this section contains some information about the work behind the drafting of this
document, such as the amount of time spent by each of the authors, the tools used to produce all
the material and a revision history that keeps track of all the modifications.
\end{enumerate}