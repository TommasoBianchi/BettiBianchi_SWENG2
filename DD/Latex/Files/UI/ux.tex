User Interface Design is an important aspect during the developing process of a system. It has the aim to model the part of the system that is related with users, hence it must be simple and efficient in terms of accomplishing users' goals. Since users' schedule is the crucial point in \projectname~, we thought to use the calendar page as home page of the application in order to facilitate the interactions that the user should do more often. Additionally UI of \projectname~ should be concise, responsive, professional and good-looking for the purpose of satisfying all the possible type of users that could interact with the system. \\
\newline
In our charts, screens marked with * are referring to other ones that will be analysed more accurately in other diagrams. \\
We are going to present 5 different UX diagrams:
\begin{itemize}
	\item{\textbf{Registration/Login}}: UX diagram of login and registration phase that can be done also using external login providers. During the registration process users have to complete an input form inserting default locations, a preference list and a nickname. This is a crucial phase in order to guarantee the correct behaviour of all other system functionalities.
	\item{\textbf{Calendar Page}}: UX diagram of the home page seen by a user after the login process. It shows meetings, travels and gives the possibility to the user to move inside the system.
	\item{\textbf{Settings}}: UX diagram of the settings page of the system. It shows all the customizable features such as the preference list, default locations, constraints, statuses and breaks.
	\item{\textbf{Notification}}: UX diagram that shows how notifications are managed by the system and how users can interact with them.
	\item{\textbf{Manage Meeting}}: UX diagram that shows all screens and available actions about a meeting, as seen from an administrator's point of view.
\end{itemize}

\begin{figure}[h]
	\centering\includegraphics[ width=\textwidth, scale = 1]{Images/UX/UXRegistrationLogin.png}{}
	\caption{Registration and Login UX}
\end{figure}

\begin{figure}[h]
	\centering\includegraphics[ width=\textwidth, scale = 1]{Images/UX/UXCalendarPage.png}{}
	\caption{Calendar Page UX}
\end{figure}

\begin{figure}[h]
	\centering\includegraphics[ width=\textwidth, scale = 1]{Images/UX/UXSettings.png}{}
	\caption{Settings UX}
\end{figure}

\begin{figure}[h]
	\centering\includegraphics[ width=\textwidth, scale = 1]{Images/UX/UXNotifications.png}{}
	\caption{Notifications UX}
\end{figure}

\begin{figure}[h]
	\centering\includegraphics[ width=\textwidth, scale = 1]{Images/UX/UXManageMeeting.png}{}
	\caption{Manage Meeting UX}
\end{figure}
\clearpage