\begin{table}[h]	
	\centering
	\def\arraystretch{1.5}
	\begin{tabular}{|m{7cm}|m{7cm}|}
		\hline
		\textbf{Test Name}            &  Unauthorized access  \\ \hline
		\textbf{Event Flow}             &  
			\begin{enumerate}
				\item~Try to load a page inside the application without being logged in.
			\end{enumerate}
		 \\ \hline
		\textbf{Expected Output}  &  The system prevents you from doing that.   \\ \hline
		\textbf{Actual Output}       &  The system prevents you from doing that.   \\ \hline
		\textbf{Notes} & The system redirects you to the login page. \\ \hline
	\end{tabular}
	\caption{Unauthorized access}
\end{table}


\begin{table}[h]	
	\centering
	\def\arraystretch{1.5}
	\begin{tabular}{|m{7cm}|m{7cm}|}
		\hline
		\textbf{Test Name}            &  Invalid Signup  \\ \hline
		\textbf{Event Flow}             &  
		\begin{enumerate}
			\item Click on 'Registrati' button.
			\begin{enumerate}
				\item Fill with an used username, a valid email and a valid password.
				\item Fill with an new username, a used email and a valid password.
				\item Fill with an new username, a valid email and two different passwords in password and password confirmation fields.
			\end{enumerate}
		\item Click on 'Conferma' button.
		\end{enumerate} \\ \hline
		\textbf{Expected Output}  &  The system do not let you to signup for invalid credentials, staying in the same page showing errors messages.   \\ \hline
		\textbf{Actual Output}       & 
		\begin{enumerate}[label=\Alph*]
			\item The system does not let the user to signup but redirects the user to the login page without signalling errors.
			\item The system lets the user to signup and show the calendar page.
			\item The system does not let the user to signup but redirects the user to the login page without signalling errors.
		\end{enumerate}    \\ \hline
		\textbf{Notes} &  TO FIX ENUMERATIONS \\ \hline
	\end{tabular}
	\caption{Invalid Signup}
\end{table}


\begin{table}[h]	
\centering
\def\arraystretch{1.5}
\begin{tabular}{|m{7cm}|m{7cm}|}
	\hline
	\textbf{Test Name}            &  Invalid Login  \\ \hline
	\textbf{Event Flow}             &  
		\begin{enumerate}
				\item~Go to the login page.
				\item~Insert invalid data, such as non-existent username or incorrect password.
				\item~Click the login button.
		\end{enumerate}
		 \\ \hline
	\textbf{Expected Output}  &  The system prevents you from logging in.  \\ \hline
	\textbf{Actual Output}       &  The system prevents you from logging in.   \\ \hline
	\textbf{Notes} & \\ \hline
\end{tabular}
\caption{Invalid Login}
\end{table}


\begin{table}[h]	
	\centering
	\def\arraystretch{1.5}
	\begin{tabular}{|m{7cm}|m{7cm}|}
		\hline
		\textbf{Test Name}            &  Valid Signup  \\ \hline
		\textbf{Event Flow}             &  
		\begin{enumerate}
			\item Click on 'Registrati' button.
			\item Insert a non-used nickname, a non-used email and two equal passwords in password and password confirmation files.
			\item Click on 'Conferma' button.
		\end{enumerate} \\ \hline
		\textbf{Expected Output}  &  The system redirects the user to an empty calendar page.   \\ \hline
		\textbf{Actual Output}       & The system redirects the user to an empty calendar page.    \\ \hline
		\textbf{Notes} & \\ \hline
	\end{tabular}
	\caption{Valid Signup}
\end{table}


\begin{table}[h]	
\centering
\def\arraystretch{1.5}
\begin{tabular}{|m{7cm}|m{7cm}|}
	\hline
	\textbf{Test Name}            &  Valid Login  \\ \hline
	\textbf{Event Flow}             &  
		\begin{enumerate}
			\item~Go to the login page.
			\item~Insert correct data.
			\item~Click the login button.
		\end{enumerate}
	 \\ \hline
	\textbf{Expected Output}  &  The system shows you your calendar.   \\ \hline
	\textbf{Actual Output}       &  The system shows you your calendar.   \\ \hline
	\textbf{Notes} & \\ \hline
\end{tabular}
\caption{Valid Login}
\end{table}


\begin{table}[h]	
\centering
\def\arraystretch{1.5}
\begin{tabular}{|m{7cm}|m{7cm}|}
	\hline
	\textbf{Test Name}            & Appointment Valid Creation   \\ \hline
	\textbf{Event Flow}             & 
	 	\begin{enumerate}
	 	\item Login.
	 	\item Click on the button to create a new appointment.
	 	\item Insert a title, two subsequent dates.
	 	\item Click on 'ADD' button.
	 \end{enumerate} \\ \hline
	\textbf{Expected Output}  &  The system redirects you to the calendar page where the new appointment is placed.   \\ \hline
	\textbf{Actual Output}       & The system redirects you to the calendar page where the new appointment is placed.    \\ \hline
	\textbf{Notes} & A user can create an appointment without inserting a location. However this is in contradiction with requirements number 8 of the RASD document. \\ \hline
\end{tabular}
\caption{Appointment Valid Creation}
\end{table}


\begin{table}[h]	
	\centering
	\def\arraystretch{1.5}
	\begin{tabular}{|m{7cm}|m{7cm}|}
		\hline
		\textbf{Test Name}            & Appointment Invalid Creation   \\ \hline
		\textbf{Event Flow}             & 
		\begin{enumerate}
			\item Login.
			\item Click on the button to create a new appointment.
			\item Insert a title, an end date that is before a start date and a location.
			\item Click on 'ADD' button.
		\end{enumerate} \\ \hline
		\textbf{Expected Output}  &  The system do not allow you to create an appointment because of invalid dates.   \\ \hline
		\textbf{Actual Output}       & The system redirects you to the calendar page where the new appointment is placed without an ending date with a default duration of 2 hours.    \\ \hline
		\textbf{Notes} & This is in contradiction with requirements number 10 of the RASD document that prevents an appointment to have such dates. \\ \hline
	\end{tabular}
	\caption{Appointment Invalid Creation}
\end{table}


\begin{table}[h]	
\centering
\def\arraystretch{1.5}
\begin{tabular}{|m{7cm}|m{7cm}|}
	\hline
	\textbf{Test Name}            &  Appointment Update  \\ \hline
	\textbf{Event Flow}             & 
		\begin{enumerate}
			\item~Login.
			\item~Open the page of a meeting already in your calendar.
			\item~Click on the update button.
			\item~Change the start time, end time and location.
			\item~Click on the save button.
		\end{enumerate}
	\\ \hline
	\textbf{Expected Output}  &  The system shows the calendar with the updated version of the meeting.   \\ \hline
	\textbf{Actual Output}       &  The system shows the calendar with the updated version of the meeting.   \\ \hline
	\textbf{Notes} & \\ \hline
\end{tabular}
\caption{Appointment Update}
\end{table}


\begin{table}[h]	
	\centering
	\def\arraystretch{1.5}
	\begin{tabular}{|m{7cm}|m{7cm}|}
		\hline
		\textbf{Test Name}            & Appointment Delete   \\ \hline
		\textbf{Event Flow}             & 
		\begin{enumerate}
			\item Login.
			\item Click on an appointment.
			\item Click on 'Delete' button.
			\item Click on 'Yes' to confirm the deletion.
		\end{enumerate} \\ \hline
		\textbf{Expected Output}  &  The system redirects you to the calendar page where the deleted appointment do not exist anymore.  \\ \hline
		\textbf{Actual Output}       & The system redirects you to the calendar page where the deleted appointment do not exist anymore.    \\ \hline
		\textbf{Notes} & \\ \hline
	\end{tabular}
	\caption{Appointment Delete}
\end{table}


\begin{table}[h]	
\centering
\def\arraystretch{1.5}
\begin{tabular}{|m{7cm}|m{7cm}|}
	\hline
	\textbf{Test Name}            &  Two meetings in the same day generates a travel  \\ \hline
	\textbf{Event Flow}             & 
		\begin{enumerate}
			\item~Login.
			\item~Create a meeting in the morning of some day.
			\item~Create a meeting in the afternoon of the same day.
		\end{enumerate}
	  \\ \hline
	\textbf{Expected Output}  &   The system shows the calendar with the two new meetings and the travel between them.  \\ \hline
	\textbf{Actual Output}       &  The system shows the calendar with the two new meetings and the travel between them.   \\ \hline
	\textbf{Notes} &  Make sure the two meetings have reasonably near locations, otherwise travel cannot be computed.  \\ \hline
\end{tabular}
\caption{Two meetings in the same day generates a travel}
\end{table}


\begin{table}[h]	
	\centering
	\def\arraystretch{1.5}
	\begin{tabular}{|m{7cm}|m{7cm}|}
		\hline
		\textbf{Test Name}            & Create an appointment that overlaps with another already present and generate a warning  \\ \hline
		\textbf{Event Flow}             & 
		\begin{enumerate}
			\item Login.
			\item Click on the button to create a new appointment.
			\item Insert a title, a start date that is in the middle of an other appointment, a later end date and a location.
			\item Click on 'ADD' button.
		\end{enumerate} \\ \hline
		\textbf{Expected Output}  &  The system prevent the user creating a new appointment because it overlaps with an existing one and signal it to the user.  \\ \hline
		\textbf{Actual Output}       & The system redirects you to the calendar page where the deleted appointment do not exist anymore.    \\ \hline
		\textbf{Notes} & \\ \hline
	\end{tabular}
	\caption{Appointment Delete}
\end{table}


\begin{table}[h]	
\centering
\def\arraystretch{1.5}
\begin{tabular}{|m{7cm}|m{7cm}|}
	\hline
	\textbf{Test Name}            &  Two meetings in the same day generates a warning if the travel is too long  \\ \hline
	\textbf{Event Flow}             &   
		\begin{enumerate}
			\item~Login.
			\item~Create a meeting in the morning of some day.
			\item~Create a meeting in the afternoon of the same day, with a location so distant from the one of the other meeting that a travel would take more than the available time.
		\end{enumerate}
	\\ \hline
	\textbf{Expected Output}  &  The system shows the user a warning.   \\ \hline
	\textbf{Actual Output}       &   The system shows the user a warning.  \\ \hline
	\textbf{Notes} &  Meetings are kept in the calendar.   \\ \hline
\end{tabular}
\caption{Two meetings in the same day generates a warning if the travel is too long}
\end{table}


\begin{table}[h]	
	\centering
	\def\arraystretch{1.5}
	\begin{tabular}{|m{7cm}|m{7cm}|}
		\hline
		\textbf{Test Name}            & Create an appointment that covers all the break time slot and signalled it to the user.\\ \hline
		\textbf{Event Flow}             & 
		\begin{enumerate}
			\item Login.
			\item Click on the button to create a new appointment.
			\item Insert a title, a start date that before 12.00, an end date that is after 14.00 and a location.
			\item Click on 'ADD' button.
		\end{enumerate} \\ \hline
		\textbf{Expected Output}  &  The system creates the appointment and signals to the user that the break of that day is not doable anymore.  \\ \hline
		\textbf{Actual Output}       & The system creates the appointment.   \\ \hline
		\textbf{Notes} & In case the break is inhibited due to an appointment that starts before 12.00 and ends after 14.00, the warning is not signalled. When the break is inhibited by a travel that the system should plan between two subsequent appointments, a warning is generated but it is not shown to the user because it is covered by the warning that signals that the appointment has been created. \\ \hline
	\end{tabular}
	\caption{Appointment Delete}
\end{table}


\begin{table}[h]	
\centering
\def\arraystretch{1.5}
\begin{tabular}{|m{7cm}|m{7cm}|}
	\hline
	\textbf{Test Name}            &  Two meetings in the same day generates a travel with a specific travel means if that's the only available one  \\ \hline
	\textbf{Event Flow}             &   
		\begin{enumerate}
			\item~Login.
			\item~Open the preferences and disable all travel means but one.
			\item~Create a meeting in the morning of some day.
			\item~Create a meeting in the afternoon of the same day.
		\end{enumerate}
	\\ \hline
	\textbf{Expected Output}  &   The system shows the calendar with the two new meetings and the travel between them, computed with the only available travel mean.  \\ \hline
	\textbf{Actual Output}       &  The system shows the calendar with the two new meetings and the travel between them, computed with the only available travel mean.   \\ \hline
	\textbf{Notes} & \\ \hline
\end{tabular}
\caption{Two meetings in the same day generates a travel with a specific travel means if that's the only available one}
\end{table}