In this chapter we present our impression on the user experience of the tested project. These are opinions from a user point of view that we have during test and use of the system.

\subsection{Login Page}
In our opinion the login page, made mostly by the strong usage of JavaScript, is good looking even if it does not follow the colour scheme and the general looks of the calendar page found once logged into the system. \\
We found some problems in case of wrong registration: if you insert some wrong data, such as a too short password or a password that is too similar to the username, the system redirects you to the login form but loosing all informations about your inconsistent inserted data.

\subsection{Calendar Main Page}
The calendar page of the system that goes through daily, weekly and monthly views of the calendar; in spite of being similar to Google Calendar page, it is really immediate and provides everything you need. \\
For what concerns the form that allows a user to create a meeting, we found that the procedure used to insert date and the location is a little too complex: locations are not suggested, for example using Google API, although these API are used to retrieve longitude and latitude from the name of location inserted; to insert a starting and ending date for an event, you have to provide always a full date format (e.g. '2018-01-10 10:30') and after a few uses it becomes very uncomfortable.

\subsection{General Considerations}
\begin{itemize}
	\item Breaks are fixed from 12.30 to 14.30 and when one or more meetings overlap with them and make them not doable, it is signalled using a window alert. This procedure is done correctly by the system. However, in spite this is written in a requirements on the RASD document of the project, we believe that not to give the possibility to create customizable breaks is limited for a complete and satisfying use of Travlendar. In addition, the system signalled you that a break is not doable only when it actually becomes not doable and you can not check it anymore.
	\item When two appointments overlap or it is not possible to link them with a feasible travel, the system signalled it with a warning using a window alert and delete the appointment that has just create the warning. In our opinion not letting you to create this event is a strong choice, e.g. the second appointment is much more important so you want to sacrifice the former, or you want to take them both. Travlendar should have taken it into account and at least give you the possibility to choose between them afterwords or create a separate page where all this warnings, maybe together with not doable breaks, are listed.
\end{itemize}